
\chapter{Preliminares sobre funciones de varias variables}

\begin{quote}
    {\it Diversas formas de describir analíticamente curvas y superficies. Curvas y superficies de nivel. Introducción a los sistemas de coordenadas curvilíneas.}
\end{quote}

En este capítulo se hace una breve introducción a la geometría analítica tridimensional con el fin de dar interpretaciones geométricas y físicas de las funciones de varias variables. Se consideran las diversas formas (explícita, implícita y paramétrizada) de describir curvas y superficies, nociones que de momento se manejan en un sentido intuitivo, y también se introducen los sistemas de coordenadas curvilíneas usuales (polares en el plano, cilíndricas y esféricas en el espacio).

Esta introducción permitirá presentar desde un punto de vista geométrico algunos de los problemas que se abordan en el cálculo diferencial y el cálculo integral de funciones de varias variables: existencia de planos tangentes a superficies, problemas de optimización (con y sin restricciones), existencia de inversas locales, definición implícita de funciones, cálculo de áreas, volúmenes y longitudes de curvas.

\section{Introducción}

El objetivo del curso es el estudio de las funciones vectoriales de varias variables reales, es decir, funciones $\f:\Omega\longrightarrow\RR^m$ definidas en un abierto $\Omega\subset\RR^n$. En lo que sigue $\x$ denotará un elemento genérico de $\RR^n$ de componentes $(x_1,\dots,x_n)$.

En bastantes cuestiones el hecho de que sea $m>1$ no involucra dificultades realmente significativas pues frecuentemente el estudio de la función

$$\f(\x)=\left(f_1(x_1,\dots,x_n),\dots,f_m(x_1,\dots,x_m)\right)$$
se reduce al de sus componenetes $f_1(\x),\dots,f_m(\x)$. Si $n=2$, (resp. $n=3$) en lugar de $\f(x_1,x_2)$ (resp. $\f(x_1,x_2,x_3)$) se suele escribir $\f(x,y)$ (resp. $\f(x,y,z)$).

Con el fin de motivar el estudio de las funciones vectoriales de varias variables conviene empezar comentando los diferentes tipos de representación geométrica que admiten estas funciones, según los valores de $n$ y $m$, lo que permitirá interpretaciones geométricas ilustrativas de los conceptos que se vayan introduciendo. Con este fin conviene comenzar utilizando las nociones de curva y superficie en un sentido intuitivo, mostrando ejemplos concretos de estos objetos geométricos que más adelante se definirán de manera precisa. Uno de los objetivos de este curso es el de dar definiciones matemáticamente rigurosas de estas nociones. Mientras tanto utilizaremos los términos ``curva'' y ``superficie'' entre comillas para indicar que estamos considerando estos conceptos desde un punto de vista intuitivo completamente informal. Comenzamos con el caso $n=1$ donde las interpretaciones geométricas son de distinta naturaleza que en el caso $n\geq 2$.