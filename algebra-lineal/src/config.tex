% Uso de UTF-8 para la codificación de caracteres
\usepackage[utf8]{inputenc}

% Configuración para idioma español y T1 para la codificación de fuentes
\usepackage[T1]{fontenc}
\usepackage[spanish]{babel}

% Para funcionalidades sobre listas
\usepackage{enumitem}

% Paquetes matemáticos
\usepackage{amsmath, amsthm, amssymb, amsfonts}
% Considerar agregar 'mathtools' para más herramientas matemáticas
% \usepackage{mathtools}

% Para hiperenlaces en el documento
\usepackage{hyperref}
\hypersetup{pdfstartview=FitH}

% Para encabezados y pie de página
\usepackage[automark]{scrlayer-scrpage}

% Ajustar la altura del encabezado
\setlength{\headheight}{15pt}

% Para incluir imágenes
\usepackage{graphicx}


% Configuración de estilo de página
\pagestyle{scrheadings}

% Configuraciones y comandos personalizados
\clubpenalty=10000
\renewcommand{\shorthandsspanish}{}
\renewcommand*{\deg}{\normalfont\text{gr}\hspace*{1mm}}
\renewcommand{\Im}{\operatorname{\rm Im}}
\newcommand{\Log}{\operatorname{\rm Log}}
\renewcommand{\Re}{\operatorname{\rm Re}}
\newcommand{\inter}{\operatorname{\rm int}}
\newcommand{\adh}[1]{\overline{#1}}
\newcommand{\fr}[1]{\partial{#1}}
\renewcommand{\vec}[1]{\mathbf{#1}}
\newcommand{\prodesc}[2]{\left\langle #1,#2\right\rangle}
\newcommand{\norm}[1]{\left\| #1\right\|}
\newcommand{\mcm}[1]{\operatorname{\rm mcm}\left(#1\right)}
\newcommand{\mcd}[1]{\operatorname{\rm mcd}\left(#1\right)}
\newcommand{\cl}[1]{\left[#1\right]}
\newcommand{\set}[1]{\left\lbrace#1\right\rbrace}
\newcommand{\suc}[1]{\left\lbrace#1\right\rbrace}
\newcommand{\abs}[1]{\left| #1\right|}
\newcommand{\ent}[1]{\left\lfloor #1\right\rfloor}
\newcommand{\eps}{\varepsilon}

% Entornos para teoremas, lemas, etc.
\newtheorem{teo}{Teorema}[chapter]
\newtheorem{prop}{Proposición}[chapter]
\newtheorem{lem}{Lema}[chapter]
\newtheorem{cor}{Corolario}[chapter]
\newtheorem{ejer}{Ejercicio}[chapter]
\theoremstyle{definition}
\newtheorem{df}{Definición}[chapter]
\newtheorem{ob}{Observación}[chapter]
\newtheorem{obs}{Observaciones}[chapter]
\newtheorem{ej}{Ejemplo}[chapter]
\newtheorem{ejs}{Ejemplos}[chapter]
\newtheorem{nota}{Nota}[chapter]

% Definiciones de conjuntos numéricos
\def\NN{\mathbb{N}}
\def\ZZ{\mathbb{Z}}
\def\QQ{\mathbb{Q}}
\def\RR{\mathbb{R}}
\def\CC{\mathbb{C}}
\def\PP{\mathbb{P}}
\def\RP{\RR\PP}
\def\F{\mathscr{F}}
\def\G{\mathscr{G}}
\def\D{\Delta}
\def\0{\vec{0}}
\def\a{\vec{a}}
\def\b{\vec{b}}
\def\u{\vec{u}}
\def\p{\vec{p}}
\def\t{\vec{t}}
\def\x{\vec{x}}
\def\y{\vec{y}}
\def\z{\vec{z}}
\def\f{\vec{f}}
\def\g{\vec{g}}

% Configuraciones de dimensiones de página (ajustar si es necesario)
\addtolength{\textwidth}{4cm}
\addtolength{\hoffset}{-2cm}
\addtolength{\textheight}{4cm}
\addtolength{\voffset}{-2cm}
