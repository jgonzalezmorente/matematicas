\chapter{Anillos}
\section{Ideales y anillos cociente}

\begin{teo}[\textbf{Teorema de la Correspondencia}] Si $I$ es un ideal de un anillo $A$, las asignaciones $J\mapsto J/I$ y $X\mapsto\pi^{-1}(X)$ definen aplicaciones biyectivas (una inversa de la otra) que conservan la inclusión entre el conjunto de los ideales de $A$ que contienen al $I$ y el conjunto de los ideales de $A/I$.
\end{teo}

\begin{proof} \
    \begin{enumerate}[label=(\arabic*)]
        \item Si $J$ es un ideal de $A$ que contiene a $I$ entonces $J/I$ es un ideal de $A/I$ y $\pi^{-1}(J/I)=J$.
        \item Si $X$ es un ideal de $A/I$ entonces $\pi^{-1}(X)$ es un ideal de $A$ que contiene a $I$ y $\pi^{-1}(X)/I=X$.
        \item Si $J\subseteq K$ son ideales de $A$ que contienen a $I$ entonces entonces $J/I\subseteq K/I$.
        \item Si $X\subseteq Y$ son ideales de $A/I$ entonces $\pi^{-1}(X)\subseteq\pi^{-1}(Y)$.
    \end{enumerate}
\end{proof}

\section{Operaciones con ideales}

Sea $A$ un anillo. Recordemos que $X$ es un subconjunto de $A$ entonces llamamos {\it ideal} de $A$ {\it generado} por $X$ al menor ideal de $A$ que contiene a $X$ y que 
$$(X) =\left\lbrace \sum_{i=1}^na_ix_i : n\geq 0, a_i\in A, x_i\in X \right\rbrace$$
Es fácil ver que la intersección de una familia de ideales de $A$ es un ideal de $A$. Eso implica que $(X)$ es también la intersección de todos los ideales de $A$ que contienen a $X$.

Si $I$ y $J$ son dos ideales de $A$ entonces la suma y el producto de $A$ son los conjuntos
\begin{equation*}
    \begin{split}
        I + J &= \left\lbrace x+y : x\in, y\in J \right\rbrace \\
        IJ    &= \left\lbrace x_1y_1 + \cdots x_ny_n : x_1\cdots,x_n\in I, y_1,\cdots,y_n\in J \right\rbrace
    \end{split}
\end{equation*}

Más generalmente, si $I_1,\dots,I_n$ son ideales, entonces la suma de estos ideales es
$$I_1+\cdots +I_n=\left\lbrace x_1+\cdots+x_n: x_1\in I_1,\dots, x_n\in I_n\right\rbrace$$
y el producto $I_1\cdots I_n$, es el ideal formado por las sumas de productos de la forma $x_1\cdots x_n$ donde $x_1\in I_1,\dots,x_n\in I_n$.

Aún más general, si $\left\lbrace I_x : x\in X\right\rbrace$ es una familia de ideales de $A$ entonces
$$\sum_{x\in X}I_x=\left\lbrace \sum_{x\in X}a_x : a_x \in I_x\text{ para todo }x\in X\text{ y }a_x=0\text{ para casi todo }x\in X\right\rbrace$$
y $\prod_{x\in X}I_x$ es el ideal formado por las sumas de productos de la forma $\prod_{x\in X}a_x$ donde $a_x\in I_x$ para todo $x\in X$ y $a_x=1$ para casi todo $x\in X$.

\begin{prop}
    Si $\{I_x: x\in X\}$ es una familia de ideales de un anillo $A$ entonces:
    \begin{enumerate}[label=(\arabic*)]
        \item $\sum_{x\in X}I_x$ es el menor ideal de $A$ que contiene a todos los $I_x$, o sea el ideal generado por $\cup_{x\in X}I_x$.
        \item Si $I_1,\dots,I_n$ son ideales de $A$ entonces $I_1\cdots I_n$ es el menor ideal de $A$ generado por los productos $x_1\cdots x_n$ con $x_1\in I_1,\dots, x_n\in I_n$.
    \end{enumerate}
\end{prop}

\begin{ej}{\it Operaciones con ideales}
    \begin{enumerate}[label=(\arabic*)]
        \item Sean $n$ y $m$ dos números enteros y consideremos los ideales $(n)$ y $(m)$ de $\ZZ$. Claramente $(n)(m)=(nm)$. Por otro lado, $(n)\cap(m)$ está formado por los números enteros que son múltiplos de $n$ y $m$. Esos son precisamente los múltiplos del mínimo común múltiplo de $n$ y de $m$. Finalmente, $(n)+(m)$ es el menor ideal $(d)$ de $\ZZ$ que contiene a $(n)$ y $(m)$, $(d)=(n)+(m)$ si y solo si $d$ divide a $n$ y a $m$ y es múltiplo de todos los divisores comunes de $n$ y $m$. O sea, $d$ es el máximo común divisor de $n$ y $m$. En resumen:
        $$(n)(m)=(nm), \quad (n)\cap(m)=(\mcm{n,m}), \quad (n)+(m)=(\mcd{n,m})$$

        \item Consideremos ahora el anillo $\ZZ[X]$ de los polinomios con coeficientes enteros. Entonces $(2) + (X)$ está formado por los polinomios cuyo término independiente es par. Vamos a ver que este ideal no es principal. Supongamos por reducción al absurdo que $(2)+(X)=(a)$ para algún $a\in\ZZ[X]$. Entonces $2=ab$ para algún polinomio $b$, lo que que implica que $a\in\ZZ$. Además, como $a\in (2, X)$, necesariamente $a$ es par, lo que implica $X \notin (a)=(2)+(X)$, una contradicción.
    \end{enumerate}
\end{ej}

\section{ Los Teoreamas de Isomorfía y Chino de los Restos}

\begin{teo}[\textbf{Primer Teorema de Isomorfía}]
    Sea $f:A\longrightarrow B$ un homomorfismo de anillos. Entonces existe un único isomorfismo de anillos $\overline{f}: A/\Ker f\longrightarrow \Im f$

    $$\xymatrix@C=1.5cm@R=1cm{ 
        A \ar[r]^f\ar[d]_p                     & B \\ 
        A/\Ker f \ar@{-->}[r]^{\overline{f}}  & \Im f\ar[u]_i 
    }$$
    
    es decir, $i\circ \overline{f} \circ p=f$, donde $i$ es la inclusión y $p$ es la proyección. En particular,
    $$ A/\Ker f\simeq \Im f$$
\end{teo}

\begin{proof}
    Sean $K$ e $I=\Im f$. La aplicación $\overline{f}: A/K\longrightarrow I$ dada por $\overline{f}(x+K)=f(x)$ está bien definida (no depende de representantes) pues si $x+K=y+K$ entonces $x-y\in K$ y por lo tanto $f(x)-f(y)=f(x-y)=0$, es decir, $f(x)=f(y)$. Además es elemental ver que es un homomorfismo de anillos y que es suprayectiva. Para ver que es inyectiva, veamos que su nucleo es nulo. Si $x+K$ está en el núcleo de $\overline{f}$ entonces $0=\overline{f}(x+K)=f(x)$, de modo que $x\in K$ y así $x+K=0+K$. Es decir $\Ker\overline{f}=0$ y por lo tanto $f$ es inyectiva. En conclusión, $\overline{f}$ es un isomorfismo, y hace conmutativo el diagrama porque, para cada $x\in A$, se tiene
    $$i\left(\overline{f}\left(p(x)\right)\right)=\overline{f}(x+K)=f(x)$$
    En cuanto a la unicidad, supongamos que otro homomorfismo $\widehat{f}:A/K\longrightarrow I$ verifica que $i\circ \widehat{f} \circ p=f$; entonces para cada $x\in A$ se tiene $\widehat{f}(x+K)=i(\widehat{f}(p(x)))=f(x)=\overline{f}(x+K)$, y por lo tanto $\widehat{f}=\overline{f}$.
\end{proof}