\chapter{Anillos}

\section{Operaciones binarias}

Sea $X$ un conjunto. Una {\it operación binaria} en $X$ es una aplicación $*:X\times X\rightarrow X$. La imagen de $(a,b)$ la denotamos por $a*b$. Decimos que $*$ es:
\begin{itemize}
    \item {\it Conmutativa} si $x*y=y*x$ para todo $x,y \in X$.
    \item {\it Asociativa} si $x*(y*z)=(x*y)*z$ para todo $x,y,z \in X$.
\end{itemize}

Un elemento $x\in X$ se dice que es:
\begin{itemize}
    \item {\it Neutro por la izquierda (neutro por la derecha)} de $X$ con respecto a $*$ si $x*y = y$ para todo $y\in X$ ($y*x=y$ para todo $y\in X$).
    \item {\it Cancelable por la izquierda (cancelable por la derecha)} en $X$ respecto a $*$ si para cada dos elementos distintos $a$ y $b$ de $X$ se verifica $x*a\neq x*b$ ($a*x\neq b*x$).
    \item Supongamos que $e$ es un elemento neutro de $X$ con respecto a $*$. Sean $x$ e $y$ elementos de $X$. Decimos que $x$ es {\it simétrico de $y$ por la izquierda} y que $y$ es {\it simétrico de $x$ por la derecha} con respecto a $*$ si se verifica que $x*y=e$.
\end{itemize}

Decimos que $x$ es
\begin{itemize}
    \item {\it Neutro} de $X$ con respecto a $*$ si es neutro por la izquierda y por la derecha de $X$ con respecto a $*$.
    \item {\it Cancelable} en $X$ con respecto a $*$ si es cancelable en $X$ con respecto a $*$ por los dos lados.
    \item {\it Simétrico} de $y$ con respecto a $*$ si es simétrico de $y$ con respecto a $*$ por los dos lados. En tal caso decimos que $x$ es {\it invertible} de $X$ respecto a $*$.
\end{itemize}

Un par $(X,*)$ formado por un conjunto y una operación binaria $*$ decimos que es un
\begin{itemize}
    \item {\it Semigrupo} si $*$ es asociativa.
    \item {\it Monoide} si es un semigrupo que tiene un elemento neutro con respecto a $*$.
    \item {\it Grupo} si es un monoide y todo elemento de $X$ es invertible con respecto a $*$.
    \item {\it Grupo abeliano} si es un grupo y $*$ es conmutativa.
\end{itemize}

En el futuro simplificaremos la terminología y en lugar de decir ``operación binaria'' diremos simplemente ``operación''. Por otro lado nos ahorraremos los ``con respecto a'' cuando la operación esté clara por el contexto y los ``de $X$'' o ``en $X$'' cuando el conjunto $X$ esté claro por el contexto o diremos que $e$ es neutro, neutro por un lado, inverso, invertible o cancelable en $(X,*)$.

Veamos algunos ejemplos.

\begin{ejs}{\it Operaciones}
    \begin{enumerate}[label=(\arabic*)]
        \item La suma es una operación en los conjuntos $\NN$ de los números naturales, $\ZZ^{\geq 0}$ de los enteros no negativos, $\ZZ$ de los números enteros, $\QQ$ de los números racionales, $\RR$ de los números reales y $\CC$ de los números complejos. En todos los casos se trata de una operación conmutativa y asociativa. Además $0$ es neutro. Todo elemento $a$ de $\ZZ$, $\QQ$, $\RR$ y $\CC$ es cancelable y es invertible con respecto a la suma y su simétrico es su opuesto $-a$. Por tanto $(\NN, +)$ es un semigrupo conmutativo, $(\ZZ^{\geq 0}, +)$ es un monoide conmutativo, y $(\ZZ, +)$, $(\QQ, +)$, $(\RR, +)$ y $(\CC, +)$ son grupos abelianos.
        \item Otra operación conmutativa y asociativa en $\NN$, $\ZZ^{\geq 0}$, $\QQ$, $\RR$ y $\CC$ es el producto. En este caso el $1$ es el neutro y todo elemento $a\neq 0$ de $\QQ$, $\RR$ y $\CC$ es invertible y su simétrico es su inverso $a^{-1}$. Sin embargo, en $\ZZ$ solamente $1$ y $-1$ son invertibles respecto del producto mientras que $1$ es el único elemento invertible de $\NN$ y $\ZZ^{\geq 0}$. Por tanto, el producto define en todos estos conjuntos una estructura de monoide conmutativo y define una estructura de grupo abeliano en $\QQ-\conj{0}$, $\RR-\conj{0}$ y $\CC-\conj{0}$. El único elemento de estos conjuntos que no es cancelativo con respecto al producto es el cero.
        \item Sea $A$ un conjunto y sea $X=A^A$, el conjunto de las aplicaciones de $A$ en $A$. La composición de aplicaciones define una operación asociativa en $X$ para la que la identidad $1_X$ es neutro. Por tanto, $(A^A, \circ)$ es un monoide. Sin embargo, esta operación no es conmutativa si $A$ tiene al menos dos elementos.
        \item Sea $A$ un conjunto y sea $X=\RR^A$ el conjunto de las aplicaciones de $A$ en $\RR$. Definimos la suma en $X$ poniendo
        $$(f+g)(a)=f(a)+g(a), \quad a\in A$$
        Esta es una operación conmutativa y asociativa, la aplicación $0$ dada por $0(a)=0$ para todo $a\in A$ es un neutro y para toda aplicación $f:A\rightarrow\RR$, el simétrico de $f$ con respecto a $+$ es la aplicación $-f$ dada por $(-f)(a)=-f(a)$. Por tanto, $(\RR^A,+)$ es un grupo abeliano.

        Definimos ahora el producto $\cdot$ en $X$ poniendo
        $$(f\cdot g)(a)=f(a)g(a), \quad a\in A$$
        Esta operación también es conmutativa y asociativa y tiene por neutro la aplicación $1$ dada por $1(a)=1$ para todo $a\in A$. Para que un elemento $f$ de $X$ sea invertible es necesario y suficiente que $f(a)\neq 0$ para todo $a\in A$. En tal caso el simétrico de $f$ con respecto a $\cdot$ es la aplicación $g$ dada por $g(a)=g(a)^{-1}$. Luego $(\RR^A, \cdot)$ es un monoide conmutativo.
    \end{enumerate}
\end{ejs}

Veamos ahora algunas propiedades básicas de las definiciones dadas más arriba.

\begin{prop}\label{anillos:prop:1}
    Sea $*$ una operación en un conjunto $X$.
    \begin{enumerate}[label=(\arabic*)]
        \item Si $*$ es conmutativa entonces todo neutro por un lado, es neutro, todo elemento cancelativo por un lado es cancelativo y todo elemento que tenga simétrico por un lado es invertible.
        \item\label{anillos:enum:1} Si $e$ es un neutro por la izquierda y $f$ es un neutro por la derecha de $X$ con respecto a $*$ entonces $e=f$. En particular, $X$ tiene a lo sumo un neutro.
        \item Supongamos que $(X, *)$ es un monoide y sea $a\in X$.
        \begin{enumerate}[label=(\alph*)]
            \item Si $x$ es simétrico por la izquierda de $a$ e $y$ es un simétrico por la derecha de $a$ entonces $x=y$.
            \item Si $a$ tiene un simétrico por un lado entonces es cancelable por ese mismo lado. En particular todo elemento invertible es cancelable.
        \end{enumerate}
    \end{enumerate}
\end{prop}

\begin{proof} \
    \begin{enumerate}[label=(\arabic*)]
        \item Es obvio.
        \item Como $e$ es neutro por la izquierda y $f$ es neutro por la derecha tenemos
        $$f=e*f=e$$
        \item[(3a)] Ahora suponemos que $(X,*)$ es un monoide. Por \ref{anillos:enum:1}, $(X,*)$ tiene un único neutro que vamos a denotar por $e$. Como $x$ es inverso por la izquierda de $a$ e $y$ es inverso por la derecha de $a$, usando la propiedad asociativa, tenemos que
        $$y=e*y=(x*a)*y=x*(a*y)=x*e=x$$
        \item[(3b)] Supongamos que $a$ es un elemento de $X$ que tiene un inverso por la izquierda $b$ y que $a*x=a*y$ para $x,y\in X$. Usando la asociatividad una vez más concluimos que
        $$x=e*x=(b*a)*x=b*(a*x)=b*(a*y)=(b*a)*y=e*y=y$$        
    \end{enumerate}
\end{proof}

Por la proposición \ref{anillos:prop:1}, si $(X,*)$ es un monoide cada elemento invertible $a$ sólo tiene un simétrico que habitualmente se denota por $a^{-1}$.



\section{Ideales y anillos cociente}

\begin{teo}[\textbf{Teorema de la Correspondencia}]\label{anillos:teo:1} Si $I$ es un ideal de un anillo $A$, las asignaciones $J\mapsto J/I$ y $X\mapsto\pi^{-1}(X)$ definen aplicaciones biyectivas (una inversa de la otra) que conservan la inclusión entre el conjunto de los ideales de $A$ que contienen al $I$ y el conjunto de los ideales de $A/I$.
\end{teo}

\begin{proof} \
    \begin{enumerate}[label=(\arabic*)]
        \item Si $J$ es un ideal de $A$ que contiene a $I$ entonces $J/I$ es un ideal de $A/I$ y $\pi^{-1}(J/I)=J$.
        \item Si $X$ es un ideal de $A/I$ entonces $\pi^{-1}(X)$ es un ideal de $A$ que contiene a $I$ y $\pi^{-1}(X)/I=X$.
        \item Si $J\subseteq K$ son ideales de $A$ que contienen a $I$ entonces entonces $J/I\subseteq K/I$.
        \item Si $X\subseteq Y$ son ideales de $A/I$ entonces $\pi^{-1}(X)\subseteq\pi^{-1}(Y)$.
    \end{enumerate}
\end{proof}

\section{Operaciones con ideales}

Sea $A$ un anillo. Recordemos que $X$ es un subconjunto de $A$ entonces llamamos {\it ideal} de $A$ {\it generado} por $X$ al menor ideal de $A$ que contiene a $X$ y que 
$$(X) =\left\lbrace \sum_{i=1}^na_ix_i : n\geq 0, a_i\in A, x_i\in X \right\rbrace$$
Es fácil ver que la intersección de una familia de ideales de $A$ es un ideal de $A$. Eso implica que $(X)$ es también la intersección de todos los ideales de $A$ que contienen a $X$.

Si $I$ y $J$ son dos ideales de $A$ entonces la suma y el producto de $A$ son los conjuntos
\begin{equation*}
    \begin{split}
        I + J &= \left\lbrace x+y : x\in, y\in J \right\rbrace \\
        IJ    &= \left\lbrace x_1y_1 + \cdots x_ny_n : x_1\cdots,x_n\in I, y_1,\cdots,y_n\in J \right\rbrace
    \end{split}
\end{equation*}

Más generalmente, si $I_1,\dots,I_n$ son ideales, entonces la suma de estos ideales es
$$I_1+\cdots +I_n=\left\lbrace x_1+\cdots+x_n: x_1\in I_1,\dots, x_n\in I_n\right\rbrace$$
y el producto $I_1\cdots I_n$, es el ideal formado por las sumas de productos de la forma $x_1\cdots x_n$ donde $x_1\in I_1,\dots,x_n\in I_n$.

Aún más general, si $\left\lbrace I_x : x\in X\right\rbrace$ es una familia de ideales de $A$ entonces
$$\sum_{x\in X}I_x=\left\lbrace \sum_{x\in X}a_x : a_x \in I_x\text{ para todo }x\in X\text{ y }a_x=0\text{ para casi todo }x\in X\right\rbrace$$
y $\prod_{x\in X}I_x$ es el ideal formado por las sumas de productos de la forma $\prod_{x\in X}a_x$ donde $a_x\in I_x$ para todo $x\in X$ y $a_x=1$ para casi todo $x\in X$.

\begin{prop}
    Si $\{I_x: x\in X\}$ es una familia de ideales de un anillo $A$ entonces:
    \begin{enumerate}[label=(\arabic*)]
        \item $\sum_{x\in X}I_x$ es el menor ideal de $A$ que contiene a todos los $I_x$, o sea el ideal generado por $\cup_{x\in X}I_x$.
        \item Si $I_1,\dots,I_n$ son ideales de $A$ entonces $I_1\cdots I_n$ es el menor ideal de $A$ generado por los productos $x_1\cdots x_n$ con $x_1\in I_1,\dots, x_n\in I_n$.
    \end{enumerate}
\end{prop}

\begin{ej}{\it Operaciones con ideales}
    \begin{enumerate}[label=(\arabic*)]
        \item Sean $n$ y $m$ dos números enteros y consideremos los ideales $(n)$ y $(m)$ de $\ZZ$. Claramente $(n)(m)=(nm)$. Por otro lado, $(n)\cap(m)$ está formado por los números enteros que son múltiplos de $n$ y $m$. Esos son precisamente los múltiplos del mínimo común múltiplo de $n$ y de $m$. Finalmente, $(n)+(m)$ es el menor ideal $(d)$ de $\ZZ$ que contiene a $(n)$ y $(m)$, $(d)=(n)+(m)$ si y solo si $d$ divide a $n$ y a $m$ y es múltiplo de todos los divisores comunes de $n$ y $m$. O sea, $d$ es el máximo común divisor de $n$ y $m$. En resumen:
        $$(n)(m)=(nm), \quad (n)\cap(m)=(\mcm{n,m}), \quad (n)+(m)=(\mcd{n,m})$$

        \item Consideremos ahora el anillo $\ZZ[X]$ de los polinomios con coeficientes enteros. Entonces $(2) + (X)$ está formado por los polinomios cuyo término independiente es par. Vamos a ver que este ideal no es principal. Supongamos por reducción al absurdo que $(2)+(X)=(a)$ para algún $a\in\ZZ[X]$. Entonces $2=ab$ para algún polinomio $b$, lo que que implica que $a\in\ZZ$. Además, como $a\in (2, X)$, necesariamente $a$ es par, lo que implica $X \notin (a)=(2)+(X)$, una contradicción.
    \end{enumerate}
\end{ej}

\section{Los Teoreamas de Isomorfía y Chino de los Restos}

\begin{teo}[\textbf{Primer teorema de isomorfía}]
    Sea $f:A\rightarrow B$ un homomorfismo de anillos. Entonces existe un único isomorfismo de anillos $\overline{f}: A/\Ker f\rightarrow \Im f$ que hace conmutativo el diagrama

    $$\xymatrix@C=1.5cm@R=1cm{ 
        A \ar[r]^f\ar[d]_p                     & B \\ 
        A/\Ker f \ar@{-->}[r]^{\overline{f}}  & \Im f\ar[u]_i 
    }$$
    
    es decir, $i\circ \overline{f} \circ p=f$, donde $i$ es la inclusión y $p$ es la proyección. En particular,
    $$ A/\Ker f\simeq \Im f$$
\end{teo}

\begin{proof}
    Sean $K=\Ker f$ e $I=\Im f$. La aplicación $\overline{f}: A/K\rightarrow I$ dada por $\overline{f}(x+K)=f(x)$ está bien definida (no depende de representantes) pues si $x+K=y+K$ entonces $x-y\in K$ y por lo tanto $f(x)-f(y)=f(x-y)=0$, es decir, $f(x)=f(y)$. Además es elemental ver que es un homomorfismo de anillos y que es suprayectiva. Para ver que es inyectiva, veamos que su nucleo es nulo. Si $x+K$ está en el núcleo de $\overline{f}$ entonces $0=\overline{f}(x+K)=f(x)$, de modo que $x\in K$ y así $x+K=0+K$. Es decir $\Ker\overline{f}=0$ y por lo tanto $f$ es inyectiva. En conclusión, $\overline{f}$ es un isomorfismo, y hace conmutativo el diagrama porque, para cada $x\in A$, se tiene
    $$i\left(\overline{f}\left(p(x)\right)\right)=\overline{f}(x+K)=f(x)$$
    En cuanto a la unicidad, supongamos que otro homomorfismo $\widehat{f}:A/K\longrightarrow I$ verifica que $i\circ \widehat{f} \circ p=f$; entonces para cada $x\in A$ se tiene $\widehat{f}(x+K)=i(\widehat{f}(p(x)))=f(x)=\overline{f}(x+K)$, y por lo tanto $\widehat{f}=\overline{f}$.
\end{proof}

\begin{teo}[\textbf{Segundo teorema de isomorfía}] Sea $A$ un anillo y sean $I$ y $J$ dos ideales tales $I\subseteq J$.
    Entonces $J/I$ es ideal de $A/I$ y existe un isomorfismo de anillos
    $$\frac{A/I}{J/I}\simeq A/J$$
\end{teo}

\begin{proof}
    Por el teorema de la correspondencia \ref{anillos:teo:1}, $J/I$ es un ideal de $A/I$. Sea $f:A/I\rightarrow A/J$ la aplicación definida por $f(a+I)=a+J$. Es elemental ver que $f$ está bien definida, que es un homomorfismo suprayectivo de anillos y que $\Ker f = J/I$. Entonces el isomorfismo buscado se obtiene aplicando el primer teorma de isomorfía.
\end{proof}

\begin{teo}[\textbf{Tercer teorema de isomorfía}] Sea $A$ un anillo con un subanillo $B$ y un ideal $I$. Entonces:
    \begin{enumerate}[label=(\arabic*)]
        \item $B\cap I$ es un ideal de $B$.
        \item $B+I$ es un subanillo de $A$ que contiene a $I$ como ideal.
        \item Se tiene el isomorfismo de anillos
        $$\frac{B}{B\cap I}\simeq \frac{B+I}{I}$$
    \end{enumerate}    
\end{teo}

\begin{proof}
    Los dos primeros apartados se dejan como ejercicio. En cuanto al último, sea $f:B\rightarrow A/I$ la composición de la inclusión $j:B\rightarrow A$ con la proyección $p:A\rightarrow A/I$. Es claro que $\Ker f=B\cap I$ y que $\Im f=(B+I)/I$, por lo que el resultado se sigue del primer teorema de isomorfía.
\end{proof}