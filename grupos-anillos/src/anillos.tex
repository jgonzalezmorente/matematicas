\chapter{Anillos}
\section{Ideales y anillos cociente}

\begin{teo}[\textbf{Teorema de la Correspondencia}] Si $I$ es un ideal de un anillo $A$, las asignaciones $J\mapsto J/I$ y $X\mapsto\pi^{-1}(X)$ definen aplicaciones biyectivas (una inversa de la otra) que conservan la inclusión entre el conjunto de los ideales de $A$ que contienen al $I$ y el conjunto de los ideales de $A/I$.
\end{teo}

\begin{proof} \
    \begin{enumerate}[label=(\arabic*)]
        \item Si $J$ es un ideal de $A$ que contiene a $I$ entonces $J/I$ es un ideal de $A/I$ y $\pi^{-1}(J/I)=J$.
        \item Si $X$ es un ideal de $A/I$ entonces $\pi^{-1}(X)$ es un ideal de $A$ que contiene a $I$ y $\pi^{-1}(X)/I=X$.
        \item Si $J\subseteq K$ son ideales de $A$ que contienen a $I$ entonces entonces $J/I\subseteq K/I$.
        \item Si $X\subseteq Y$ son ideales de $A/I$ entonces $\pi^{-1}(X)\subseteq\pi^{-1}(Y)$.
    \end{enumerate}
\end{proof}

\section{Operaciones con ideales}

Sea $A$ un anillo. Recordemos que $X$ es un subconjunto de $A$ entonces llamamos {\it ideal} de $A$ {\it generado} por $X$ al menor ideal de $A$ que contiene a $X$ y que 
$$(X) =\left\lbrace \sum_{i=1}^na_ix_i : n\geq 0, a_i\in A, x_i\in X \right\rbrace$$
Es fácil ver que la intersección de una familia de ideales de $A$ es un ideal de $A$. Eso implica que $(X)$ es también la intersección de todos los ideales de $A$ que contienen a $X$.

Si $I$ y $J$ son dos ideales de $A$ entonces la suma y el producto de $A$ son los conjuntos
\begin{equation*}
    \begin{split}
        I + J &= \left\lbrace x+y : x\in, y\in J \right\rbrace \\
        IJ    &= \left\lbrace x_1y_1 + \cdots x_ny_n : x_1\cdots,x_n\in I, y_1,\cdots,y_n\in J \right\rbrace
    \end{split}
\end{equation*}

Más generalmente, si $I_1,\dots,I_n$ son ideales, entonces la suma de estos ideales es
$$I_1+\cdots +I_n=\left\lbrace x_1+\cdots+x_n: x_1\in I_1,\dots, x_n\in I_n\right\rbrace$$
y el producto $I_1\cdots I_n$, es el ideal formado por las sumas de productos de la forma $x_1\cdots x_n$ donde $x_1\in I_1,\dots,x_n\in I_n$.

Aún más general, si $\left\lbrace I_x : x\in X\right\rbrace$ es una familia de ideales de $A$ entonces
$$\sum_{x\in X}I_x=\left\lbrace \sum_{x\in X}a_x : a_x \in I_x\text{ para todo }x\in X\text{ y }a_x=0\text{ para casi todo }x\in X\right\rbrace$$
y $\prod_{x\in X}$ es el ideal formado por las sumas de productos de la forma $\prod_{x\in X}a_x$ donde $a_x\in I_x$ para todo $x\in X$ y $a_x=1$ para casi todo $x\in X$.

\begin{prop}
    Si $\{I_x: x\in X\}$ es una familiade ideales de un anillo $A$ entonces:
    \begin{enumerate}[label=(\arabic*)]
        \item $\sum_{x\in X}I_x$ es el menor ideal de $A$ que contiene a todos los $I_x$, o sea el ideal generado por $\cup_{x\in X}I_x$.
        \item Si $I_1,\dots,I_n$ son ideales de $A$ entonces $I_1\cdots I_n$ es el menor ideal de $A$ generado por los productos $x_1\cdots x_n$ con $x_1\in I_1,\dots, x_n\in I_n$.
    \end{enumerate}
\end{prop}