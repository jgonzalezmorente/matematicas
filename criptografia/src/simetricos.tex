
\chapter{Criptosistemas simétricos o de clave privada}

\section {Criptografía}

La {\it criptografía} es la ciencia de representar información de forma opaca para que sólo los agentes autorizados (personas o dispositivos diversos) sean capaces de desvelar el mensaje oculto. El proceso de ocultar la información se llama {\it cifrado}, pero a menudo también se llama {\it encriptado} por influencia del inglés. El proceso de desvelarla se llama {\it descifrado} o {\it desencriptado}. El concepto de criptosistema modela los procesos de cifrado y descifrado.

Un {\it criptosistema simétrico}, también llamado {\it de clave privada}, está formado por un conjunto $K$, cuyos elementos llamamos {\it claves} o {\it llaves}, y una regla que asocia dos aplicaciones a cada clave $k\in K$:

$$c_k:M_k\longrightarrow C_k, \quad d_k:C_k\longrightarrow M_k$$
de forma que 
$$d_k(c_k(x))=x, \text{ para todo } x\in M_k$$ 
En la práctica el protocolo criptográfico también incluye un algoritmo generador de claves, es decir, uno que tiene como salida un elemento de $K$, pero nosotros no vamos a tener en cuenta esta parte del criptosistema.

Extendemos las aplicaciones $c_k$ a $M_k^{\infty}=\cup_{n\geq 1}M_k^n$ y $d_k$ a $C_k^{\infty}=\cup_{n\geq 1}C_k^n$ poniendo
\begin{equation*}
    \begin{split}
        c_k(x_1\dots x_n) &= c_k(x_1)\cdots c_k(x_n), \quad x_1,\dots,x_n\in M_k \\
        d_k(y_1\dots y_n) &= d_k(y_1)\cdots d_k(y_n), \quad y_1,\dots,y_n\in C_k \\
    \end{split}
\end{equation*}

Obsérvese que representamos los elementos de $M^n$ como concatenación de elementos de $M$.

Utilizaremos la siguiente terminlogía para una clave $k\in K$:

\begin{itemize}[label=--]
    \item Elementos de $M_k^\infty$: {\it Mensajes en claro}.
    \item Elementos de $C_k^\infty$: {\it Mensajes en cifrados o encriptados}.
    \item Elementos de $c_k$: {\it Función de cifrado o función de encriptado}.
    \item Elementos de $d_k$: {\it Función de descifrado o función de desencriptado}.
\end{itemize}

Un mensaje en claro o cifrado diremos que es {\it básico} si tiene longitud $1$. Sin embargo, en muchas situaciones diremos mensajes para referirnos a mensajes básicos, bien en claro o cifrados. 

Frecuentemente $M_k$ es el mismo conjunto para todas las claves y lo mismo ocurre con los $C_k$. En este caso ponemos $M=M_k$ y $C=C_k$.

\begin{ej}\label{ej1} {\it Criptosistemas}.
    \\
    Sea $A$ un conjunto finito y denotemos por $S_A$ el conjunto de permutaciones de los elementos de $A$.
    \begin{enumerate}[label=(\arabic*)]
        \item {\it Sustitución.} Tomamos como conjunto de claves $K=S_A$, como conjuntos de mensajes básicos $M=C=A$ y como funciones de cifrado y descifrado:
        $$c_\sigma(x)=\sigma(x)\quad\text{ y }\quad d_\sigma(y)=\sigma^{-1}(y)$$
        \item {\it Reordenamiento.} Ponemos $K=\cup_{n\geq 2}S_n$, donde $S_n=S_{\{1,\dots,n\}}$. Si la clave $\sigma$ está en $S_n$ entonces ponemos $M_\sigma=C_\sigma=A^n$ y ciframos reordenando las posiciones de los símbolos de los mensajes en claro. Más precisamente:
        $$c_\sigma(x_1\cdots x_n)=x_{\sigma(1)}\cdots x_{\sigma(n)}\quad\text{ y }\quad d_\sigma(y_1\cdots y_n)=x_{\sigma^{-1}(1)}\cdots x_{\sigma^{-1}(n)}$$
    \end{enumerate}
\end{ej}

En la práctica casi todos los protocolos criptográficos son combinaciones de los del ejemplo \ref{ej1}. Sin embargo, en estos criptosistemas no se dice nada sobre las permutaciones elegidas. La naturaleza de estas permutaciones es lo que hace un criptosistema bueno o malo. La bondad de un criptosistema depende de que satisfaga las siguientes condiciones:

\begin{itemize}[label=$\bullet$]
    \item {\it Rapidez de los cálculos}. Es importante disponer de un algoritmo eficiente (polinomial, con exponente pequeño) para calcular $c_k(x)$ y $d_k(y)$.
    \item {\it Seguridad}. Debe ser difícil descubrir un valor concreto de $x$ a partir del valor de $c_k(x)$ sin conocer $k$.
\end{itemize}

Las nociones de ``algoritmo eficiente'', ``difícil de calcular'' y ``tiempo razonable'' son ambiguas. Más adelante daremos conceptos más precisos. De momento, veamos ejemplos concretos.

\section{Entropía}

\begin{teo}
    Una función continua definida sobre el conjunto de funciones de distribución de longitud $n$ que cumpla las condiciones:

    \begin{enumerate}[label=(\arabic*)]
        \item\label{item:l1-i1} $H\left(\dfrac{1}{n},\dots,\dfrac{1}{n}\right) < H\left(\dfrac{1}{n+1},\dots,\dfrac{1}{n+1}\right)$
        \item\label{item:l1-i2} $\displaystyle H\left(\dfrac{1}{n},\dots,\dfrac{1}{n}\right) = H\left(\dfrac{k_1}{n},\dots,\dfrac{k_m}{n}\right)+\sum_{i=1, k_i\neq 0}^m\dfrac{k_i}{n}H\left(\dfrac{1}{k_i},\dots,\dfrac{1}{k_i}\right)$ siempre que $\displaystyle \sum_{i=1}^mk_i=n$
    \end{enumerate}
    es de la forma
    $$H(p_1,\dots,p_n)=\sum_{i=1, p_i\ne 0}^np_i\log_b\left(\dfrac{1}{p_i}\right)=-\sum_{i=1, p_i\ne 0}^np_i\log_bp_i$$
    para algún $b>1$. 
\end{teo}

\begin{proof}
    Si $m|n$, entonces
    \begin{equation*}
        \begin{split}
            H\left(\frac{1}{n},\dots,\frac{1}{n}\right) &= H\left(\frac{m}{n},\dots,\frac{m}{n}\right) + \sum_{i=1}^{n/m}\frac{m}{n}H\left(\frac{1}{m},\dots,\frac{1}{m}\right) = \\
            &= H\left(\frac{m}{n},\dots,\frac{m}{n}\right) + H\left(\frac{1}{m},\dots,\frac{1}{m}\right)
        \end{split}
    \end{equation*}
    En particular, si $n=m^s$, entonces
    $$H\left(\frac{1}{m^s},\dots,\frac{1}{m^s}\right) = H\left(\frac{1}{m^{s-1}},\dots,\frac{1}{m^{s-1}}\right) + H\left(\frac{1}{m},\dots,\frac{1}{m}\right)$$
    Sea $g(n)=H\left(\frac{1}{n},\dots,\frac{1}{n}\right)$, entonces
    $$g(m^s)=g(m^{s-1})+g(m)$$
    y por inducción sobre $s$, se obtiene que
    $$g(m^s)=sg(m)$$
    
    La condición \ref{item:l1-i1}, implica que $g$ es estrictamente creciente y, por tanto, para todo $m>1$ tenemos $g(m^s)<g(m^{s+1})$, es decir, $sg(m)<(s+1)g(m)$. Por tanto $g(m)$ es positivo.

    Sean $n$, $k$ y $m$ enteros mayores a $1$ y sea $s$ 
    $$s=\max\conj{j\in\ZZ : j\geq 0, m^j\leq n^k}$$
    entonces $m^s\leq n^k<m^{s+1}$. Como $g$ es estrictamente creceinte, $g(m^s)\leq g(n^k)\leq g(m^{s+1})$, o equivalentemente
    $$sg(m)\leq kg(n)\leq (s+1)g(m)$$
    Como $\log$ también es una función creciente también tenemos
    $$s\log(m)\leq k\log(n)\leq (s+1)\log(m)$$
    Por tanto,
    $$\frac{s}{k}\leq\frac{g(n)}{g(m)}\leq\frac{s+1}{k}\quad\text{y}\quad\frac{s}{k}\leq\frac{\log(n)}{\log(m)}\leq\frac{s+1}{k}$$
    luego
    $$\abs{\frac{g(n)}{g(m)}-\frac{\log(n)}{\log(m)}}\leq\frac{1}{k}$$
    Como $k$ es arbitrario,
    $$\frac{g(n)}{g(m)}=\frac{\log(n)}{\log(m)}$$
    es decir,
    $$\frac{g(n)}{\log(n)}=\frac{g(m)}{\log(m)}=C$$
    Luego $g(n)=C\log(n)$ para algún número postivo $C$. Por tanto, si elegimos una base $b$ adecuada, tendremos que $g(n)=\log_bn$.

    Supongamos ahora que $\left(p_1,\dots,p_k\right)$ es una distribución de probabilidad formada por números racionales. Poniéndolos con común denominador podemos suponer que $p_i=\dfrac{b_i}{n}$ y, de la propiedad \ref{item:l1-i2} tenemos,
    \begin{equation*}
        \begin{split}
            H(p_1,\dots,p_k) &= H\left(\frac{b_1}{n},\dots,\frac{b_k}{n}\right) = g(n) - \sum_{i=1, b_i\neq 0}^k\frac{b_i}{n}g(b_i) = \log_bn - \sum_{i=1,b_i\neq 0}^k\frac{b_i}{n}\log_bb_i = \\
            &= \sum_{i=1,b_i\neq 0}^k\frac{b_i}{n}\log_b\frac{n}{b_i}=\sum_{i=1,p_i\neq 0}^kp_i\log_b\frac{1}{p_i}
        \end{split}
    \end{equation*}
    Como $H$ es continua, entonces
    $$H(p_1,\dots,p_k) = \sum_{i=1,p_i\neq 0}^kp_i\log_b\frac{1}{p_i}$$
    para toda $k-\text{upla}$ $(p_1,\dots,p_k)$ de números reales en el dominio de $H$.
\end{proof}

\begin{df}
    Sea $b$ un número real mayor que $1$. Se llama {\it entropía} en base $b$ de una distribución de probabilidad $P=(p_1,\dots,p_k)$ a
    $$H_b(p_1,\dots,p_k)=\sum_{i=1}^kp_i\log_b\frac{1}{p_i}$$
    La entropía de una variable aleatoria discreta es la entropía de su distribución de probabilidad.
\end{df}

La base $b$ en la que se calcule la función de entropía sólo implica un cambio de escala debido a la igualdad $\log_bx=\log_{b'}x\cdot\log_bb'$ que implica
$$H_b(X)=H_{b'}(X)\cdot\log_bb'$$

\begin{prop}
    Sea $(p_1,\dots,p_n,q_1,\dots,q_m)$ una distribución de probabilidad. Si $a=\sum_{i=1}^{n}p_i$, con $0<a<1$ entonces
    $$H(p_1,\dots,p_n,q_1,\dots,q_m)=H(a, 1-a)+aH\left(\frac{p_1}{a},\dots,\frac{p_n}{a}\right)+(1-a)H\left(\frac{q_1}{1-a},\dots,\frac{q_m}{1-a}\right) $$    
\end{prop}

\begin{proof}
    \begin{equation*}
        \begin{split}
            H(p_1,\dots,p_n,q_1,\dots,q_m)=& \sum_{i=1}^np_i\log\frac{1}{p_i} + \sum_{i=1}^mq_i\log\frac{1}{q_i} = \sum_{i=1}^np_i\log\frac{a}{ap_i} + \sum_{i=1}^mq_i\log\frac{1-a}{(1-a)q_i} = \\
            =&\sum_{i=1}^np_i\left(\log\frac{a}{p_i}+\log\frac{1}{a}\right) + \sum_{i=1}^mq_i\left(\log\frac{1-a}{q_i}+\log\frac{1}{1-a}\right) = \\
            =&\sum_{i=1}^np_i\log\frac{a}{p_i} + \sum_{i=1}^np_i\log\frac{1}{a} + \sum_{i=1}^mq_i\log\frac{1-a}{q_i}+\sum_{i=1}^mq_i\log\frac{1}{1-a}=\\
            &=a\log\frac{1}{a}+(1-a)\log\frac{1}{1-a} + \sum_{i=1}^np_i\log\frac{a}{p_i} + \sum_{i=1}^mq_i\log\frac{1-a}{q_i} = \\
            &=H(a,1-a) + a\sum_{i=1}^n\frac{p_i}{a}\log\frac{a}{p_i} + (1-a)\sum_{i=1}^m\frac{q_i}{1-a}\log\frac{1-a}{q_i} = \\
            &= H(a, 1-a) +aH\left(\frac{p_1}{a},\dots,\frac{p_n}{a}\right) + (1-a)H\left(\frac{q_1}{1-a},\dots,\frac{q_m}{1-a}\right)
        \end{split}
    \end{equation*}
\end{proof}

Vamos ahora a ver cual es el rango de la función de entropía. Más concretamente vamos a demostrar el siguiente.

\begin{teo}\label{teo:t1}
    Sea $X$ una variable aleatoria discreta con $n$ sucesos posibles. Entonces
    $$0\leq H_b(X)\leq\log_bn$$
    Además $H_b(X)=0$ precisamente si $P(X=x)=1$ para algún suceso $x$ y $H_b(X)=\log_bn$ si y sólo si la distribución de probabilidad de $X$ es uniforme.
\end{teo}

Para demostrar el Teorema \ref{teo:t1} necesitaremos dos lemas. El primero es bien conocido:

\begin{lem}\label{lem:l1}
    Para todo número real positivo $x$ se verifica $\log x\leq x-1$ y la igualdad se verifica precisamente si $x=1$.
\end{lem}

El segundo es un poco más complicado:

\begin{lem}
    Sea $P=(p_1,\dots,p_n)$ una distribución de probabilidad y $Q=(q_1,\dots,q_n)\in\RR^n$ con $0\leq q_i\leq 1$ y $\sum_{i=1}^nq_i\leq 1$. Entonces
    $$\sum_{i=1, p_i\neq 0}^np_i\log\frac{1}{p_i}\leq\sum_{i=1, q_i\neq 0}p_i\log\frac{1}{q_i}$$
    Además la igualdad se verifica precisamente si $p_i=q_i$ para todo $i$.
\end{lem}

\begin{proof}
    Del lema \ref{lem:l1} se tiene que si $p\neq 0$ y $q\neq 0$ entonces
    $$\log\frac{q}{p}\leq\frac{q}{p}-1$$
    y, por tanto,
    $$p\log\frac{1}{p}\leq p\log\frac{1}{q} + q - p$$
    Puesto que $\sum_{i=1}^n q_i\leq 1=\sum_{i=1}^np_i$, se tiene
    $$\sum_{i=1,p_i\neq 0}^np_i\log\frac{1}{p_i}\leq \sum_{i=1,p_i\neq 0,q_i\neq 0}^n\left(p_i\log\frac{1}{q_i}+q_i-p_i\right)\leq \sum_{i=1,q_i\neq 0}^np_i\log\frac{1}{q_i}$$

    Supongamos que se da la igualdad, esto es, 
    $$\sum_{i=1, p_i\neq 0}^np_i\log\frac{1}{p_i}=\sum_{i=1, q_i\neq 0}p_i\log\frac{1}{q_i}$$
    Entonces,
    $$p_i\log\frac{1}{p_i}=p_i\log\frac{1}{q_i}+q_i-p_i$$
    para todo $i$ con $p_i\neq 0$ y $q_i\neq 0$, o equivalentemente,
    $$\log\frac{q_i}{p_i}=\frac{q_i}{p_i}-1$$
    Pero del lema \ref{lem:l1}, esto equivale a que $p_i=q_i$ para todo $i$.

\end{proof}